\let\negmedspace\undefined
\let\negthickspace\undefined
\documentclass[journal,12pt,twocolumn]{IEEEtran}
\usepackage{amssymb}
\usepackage{cite}
\usepackage{amsmath,amssymb,amsfonts,amsthm}
\usepackage{algorithmic}
\usepackage{graphicx}
\usepackage{textcomp}
\usepackage{xcolor}
\usepackage{txfonts}
\usepackage{listings}
\usepackage{enumitem}
\usepackage{mathtools}
\usepackage{gensymb}
\usepackage{comment}
\usepackage[breaklinks=true]{hyperref}
\usepackage{tkz-euclide} 
\usepackage{listings}
\usepackage{gvv}                                        
\def\inputGnumericTable{}                                 
\usepackage[latin1]{inputenc}                                
\usepackage{color}                                            
\usepackage{array}                                            
\usepackage{longtable}                                       
\usepackage{calc}                                             
\usepackage{multirow}                                         
\usepackage{hhline}                                           
\usepackage{ifthen}                                           
\usepackage{lscape}
\usepackage{pgfplots}
\newtheorem{theorem}{Theorem}[section]
\newtheorem{problem}{Problem}
\newtheorem{proposition}{Proposition}[section]
\newtheorem{lemma}{Lemma}[section]
\newtheorem{corollary}[theorem]{Corollary}
\newtheorem{example}{Example}[section]
\newtheorem{definition}[problem]{Definition}
\newcommand{\BEQA}{\begin{eqnarray}}
\newcommand{\EEQA}{\end{eqnarray}}
\newcommand{\define}{\stackrel{\triangle}{=}}
\theoremstyle{remark}
\newtheorem{rem}{Remark}
\begin{document}

\bibliographystyle{IEEEtran}
\vspace{3cm}

\title{NCERT Discrete-10.5.3-7}
\author{EE22BTECH11004 - Allu Lohith}

\maketitle
\newpage
\bigskip

\renewcommand{\thefigure}{\theenumi}
\renewcommand{\thetable}{\theenumi}
\begin{enumerate}
\item[1.]
Find the sum of the first 22 terms of an AP in which $d = 7$ and the 22nd term is 149.
\item[Ans:]
Let the series be $$a(0),a(1),a(2),a(3),\ldots,a(n)$$

\begin{table}[h!]
\centering
\begin{tabular}{|c|c|c|}
\hline 
   \textbf{Parameter}  &\textbf{Description} &\textbf{Value} \\
\hline
     $\lambda$ & Wavelength of light &\\
\hline
$y_i\brak{t}$ & Displacement produced by $S_{i^{th}}$  & \\
\hline
$\omega$ & Angular frequency & \\
\hline
$I$ & Intensity of light at $\Delta x = \lambda$ &$K$ \\
\hline
$k $ & Wave number &$ \dfrac{2\pi}{\lambda}$  \\ 
\hline
$I_{\text{net}},I_{\text{R}} $& Net Intensities of resulting waves  &$ kA^2$  \\
\hline
\multirow{3}{*}{$\Delta x = x_1 - x_2$} & \multirow{3}{*}{Path difference} & $\lambda$ \\
\cline {3-3}
& & $\dfrac{\lambda}{3}$ \\ 
\hline
$A$ & Amplitudes of light waves & $A_1=A_2$\\
\hline
\end{tabular}

\vspace{0.5cm}
\caption{\normalsize Parameters}
\end{table}

Now, the $22^{nd}$ term means $a(21)$, so

\begin{align}
a(21) &= a(0)+nd\\
149 &= a(0)+21(7)\\
a(0) &= 149-147\\
a(0) &= 2    
\end{align}

As
\begin{align}
S(n) &= \left(\frac{n+1}{2}\right) (a(0)+nd)    
\end{align}

So, 
\begin{align}
S(21)&=\left(\frac{21+1}{2}\right)(2\times2+21\times7)\\
S(21)&=11\times151\\
S(21)&=1661
\end{align}

Python code for finding the sum of terms of the AP:
\begin{lstlisting}
a=2
n=21
d=7
s=((n+1)/2)*(2*a+n*d)
print("The sum of 22 terms is ",s)
\end{lstlisting}

\begin{table}[h!]
\centering
\renewcommand{\arraystretch}{2}
\begin{tabular}{|c|p{2cm}|c|c|}
\hline 
\textbf{Parameter}  &\textbf{Description} &\textbf{Formula} &\textbf{value} \\
\hline
$\lambda_a$ & Wave length of the reflected sound & $v_a/\text{f}$& $0.34mm$  \\
\hline
$\lambda_w$ &  Wave length of the reflected sound & $v_w/\text{f}$ &$1.486mm$ \\
\hline
$K_w$ & Wavenumber of sound wave in air & $\lambda_a/2\pi$ & $54 \times 10^{-6} m^{-1}$ \\
\hline
$K_a$ & Wavenumber of sound wave in water & $\lambda_w/2\pi$ & $236 \times 10^{-6} m^{-1}$ \\
\hline
\end{tabular}

\vspace{0.5cm}
\caption{\normalsize Results}
\end{table}

By differentiation property,
\begin{align}
    n\brak{n} &\overset{z} \iff (-z)\frac{dX(z)}{dz}\\
    \implies nu(n) &\overset{z}\iff \frac{z^{-1}}{(1-z^{-1})^2}, |z|>1 \\
    \implies n^{2}u(n) &\overset{z}\iff \frac{z^{-1}\brak{z^{-1}+1}}{\brak{1-z^{-1}}^3}, |z|>1
\end{align}

For the sequence \(x(n)\), the sum of the first \(n + 1\) terms can be expressed as
\begin{align}
y(z) &= \sum_{n=-\infty}^{n=\infty} a(n)\cdot z^{-n}\\
X(z) &= \sum_{n=0}^{n=\infty}\left(\frac{n+1}{2}\right)(2\brak{0}+nd)\cdot n\cdot z^{-n}
\end{align}

\begin{align}
    X(z) &= 2x(0)U(z) - 2x(0)z\frac{dU(z)}{dz}+dz^2\frac{d^2U(z)}{dz^2} 
\end{align}
\begin{align}
\therefore X(z) &= \frac{z^{-1}\brak{d-x(0)}+x(0)}{2\brak{\brak{
1-z^{-1}}^3}}, {z\in \mathbb{C}:|z|>1} 
\end{align}
For the $(n+1)^{th}$ term in the progression,
\begin{align}
    y(n) &= y(0)+nd\\
    n\cdot u(n) &\overset{z}\iff \frac{z^{-1}}{\brak{1-z^{-1}}^{-2}}\\
    Y(z) &= \sum_{n=-\infty}^{n=\infty}y(n)\times z^{-n}\\
    Y(z) &= \sum_{n=-\infty}^{n=\infty}\brak{y(0)+nd}u(n)\cdot z^{-n}\\
    Y(z) &= y(0)U(z)-zd\frac{dU(z)}{dz}\\
    \therefore Y(z) &= \frac{y(0)+\brak{d-y(0)}z^{-1}}{\brak{1-z^{-1}}^2}, {z \in \mathbb{C}: |z|>1}
\end{align}
\end{enumerate}
\end{document}
