% \iffalse
\let\negmedspace\undefined
\let\negthickspace\undefined
\documentclass[journal,12pt,twocolumn]{IEEEtran}
\usepackage{amssymb}
\usepackage{cite}
\usepackage{amsmath,amssymb,amsfonts,amsthm}
\usepackage{algorithmic}
\usepackage{graphicx}
\usepackage{textcomp}
\usepackage{xcolor}
\usepackage{txfonts}
\usepackage{listings}
\usepackage{enumitem}
\usepackage{mathtools}
\usepackage{gensymb}
\usepackage{comment}
\usepackage[breaklinks=true]{hyperref}
\usepackage{tkz-euclide} 
\usepackage{listings}
\usepackage{gvv}                                        
\def\inputGnumericTable{}                                 
\usepackage[latin1]{inputenc}                                
\usepackage{color}                                            
\usepackage{array}                                            
\usepackage{longtable}                                       
\usepackage{calc}                                             
\usepackage{multirow}                                         
\usepackage{hhline}                                           
\usepackage{ifthen}                                           
\usepackage{lscape}
\usepackage{pgfplots}
\newtheorem{theorem}{Theorem}[section]
\newtheorem{problem}{Problem}
\newtheorem{proposition}{Proposition}[section]
\newtheorem{lemma}{Lemma}[section]
\newtheorem{corollary}[theorem]{Corollary}
\newtheorem{example}{Example}[section]
\newtheorem{definition}[problem]{Definition}
\newcommand{\BEQA}{\begin{eqnarray}}
\newcommand{\EEQA}{\end{eqnarray}}
\newcommand{\define}{\stackrel{\triangle}{=}}
\theoremstyle{remark}
\newtheorem{rem}{Remark}
\begin{document}

\bibliographystyle{IEEEtran}
\vspace{3cm}

\title{NCERT Discrete-10.5.3-7}
\author{EE22BTECH11004 - Allu Lohith}

\maketitle
\newpage
\bigskip

\renewcommand{\thefigure}{\theenumi}
\renewcommand{\thetable}{\theenumi}
\begin{enumerate}
\item[1.]
Find the sum of the first 22 terms of an AP in which $d = 7$ and the 22nd term is 149.
\item[Ans:]
Let the series be $$x(0),x(1),x(2),x(3),\ldots,x(n)$$

\begin{table}[h!]
\centering
\renewcommand{\arraystretch}{2}
\begin{tabular}{|c|c|c|}
\hline 
\setlength{\tabcolsep}{1pt}
\textbf{Parameter}  &\textbf{Description} &\textbf{Formulae/Value} \\
\hline
$a\brak 0$ & First term of A.P & - \\
\hline
\textbf{$d$} & Commom difference & - \\
\hline
n & Count of terms starting from '0' & - \\
\hline
$a\brak n$ & $(n+1)^{th}$ term of the A.P & $a\brak0 + nd$ \\
\hline
$a\brak{21}$ & Value of $22^{nd}$ term & 149 \\

\hline
$S\brak n$ & Sum of (n+1) terms in A.P & $\left(\frac{n+1}{2}\right) (2a\brak0+nd)$ \\
\hline
\end{tabular}

\vspace{0.5cm}
\caption{\normalsize Parameters}
\end{table}

Now, the $22^{nd}$ term means $a(21)$, so

\begin{align}
x(21) &= x(0)+nd\\
149 &= x(0)+21(7)\\
x(0) &= 2    
\end{align}

The general term is $x(n)=2+7d$
The z transform of the general term is 
\begin{align}
x(z)&= \frac{x(0)}{1-z^{-1}}+\frac{dz^{-1}}{\brak{1-z^{-1}}^2} \\
&=\frac{2}{1-z^{-1}}+\frac{7z^{-1}}{{\brak{1-z^{-1}}^2}}\\
&=\frac{2+5z^{-1}}{\brak{1-z^{-1}}^2}
\end{align}
On convolution for finding the sum
\begin{align}
    y(n)&=x(n)\ast u(n)\\
    \implies y(z)&=x(z)\cdot u(n)\\
    \implies y(z)&=\left[\frac{{2+5z^{-1}}}{{(1-z^{-1})^2}}\right] \cdot \frac{1}{{1-z^{-1}}}\\
    \implies y(z)&= \frac{2+5z^{-1}}{\brak{1-z^{-1}}^3}
\end{align}
Using Contour integration to find the inverse z-transform,
\begin{align}
    y(n)&=\oint_c y(z)\cdot z^{n-1}dz\\
    y(21)&=\oint_c \frac{2+5z^{-1}}{\brak{1-z^{-1}}^3}\cdot z^{20}dz
\end{align}
We can observe there are three poles and thus m = 3,
\begin{align}
    R&=\frac{1}{\brak{n-1}!} \lim_{z \to a} \frac{d^{m-1}}{dz^{m-1}}\brak{\brak{z-a}^mf\brak z}\\
    &=\frac{1}{2!}\lim_{z \to 1} \frac{d^2}{dz^2} \brak{\brak{z-1}^3\cdot \frac{2+5z^{-1}}{\brak{1-z^{-1}}^3}\cdot \brak{z^{20}}}\\
    &=\frac{1}{2}\brak{1012+2310}\\
    \implies R&=1661
\end{align}


Python code for finding the sum of terms of the AP:
\begin{lstlisting}
x=2
n=21
d=7
y=((n+1)/2)*(2*a+n*d)
print("The sum of 22 terms is ",y)
\end{lstlisting}

\begin{table}[h!]
\centering
\begin{tabular}{|c|p{2cm}|c|c|}
\hline 
\textbf{Parameter}  &\textbf{Description} &\textbf{Formula} &\textbf{value} \\
\hline
$\lambda_a$ & Wave length of the reflected sound & $v_a/\text{f}$& $0.34mm$  \\
\hline
$\lambda_w$ &  Wave length of the reflected sound & $v_w/\text{f}$ &$1.486mm$ \\
\hline
\end{tabular}

\vspace{0.5cm}
\caption{\normalsize Results}
\end{table}


\end{enumerate}
\end{document}
